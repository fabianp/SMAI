% INSTRUCTIONS :
% *a* renommer ce fichier   minisymposium-ABCDEF.tex
%        en rempla\c cant   ABCDEF   par l'acronyme de votre mini-symposium.
% *b* remplir et compl\`eter le fichier ainsi renomm\'e, suivant les indications ci-dessous.
% *c* au cours de ce travail, ne supprimer et  ne modifier aucun des macros ! merci.
% *d* compiler les fichiers SMAI-MS-Descriptions.tex  et  SMAI-MS-Resumes.tex
%        apr�s avoir remplac� dans   % INSTRUCTIONS :
% *a* renommer ce fichier   minisymposium-ABCDEF.tex
%        en rempla\c cant   ABCDEF   par l'acronyme de votre mini-symposium.
% *b* remplir et compl\`eter le fichier ainsi renomm\'e, suivant les indications ci-dessous.
% *c* au cours de ce travail, ne supprimer et  ne modifier aucun des macros ! merci.
% *d* compiler les fichiers SMAI-MS-Descriptions.tex  et  SMAI-MS-Resumes.tex
%        apr�s avoir remplac� dans   % INSTRUCTIONS :
% *a* renommer ce fichier   minisymposium-ABCDEF.tex
%        en rempla\c cant   ABCDEF   par l'acronyme de votre mini-symposium.
% *b* remplir et compl\`eter le fichier ainsi renomm\'e, suivant les indications ci-dessous.
% *c* au cours de ce travail, ne supprimer et  ne modifier aucun des macros ! merci.
% *d* compiler les fichiers SMAI-MS-Descriptions.tex  et  SMAI-MS-Resumes.tex
%        apr�s avoir remplac� dans   % INSTRUCTIONS :
% *a* renommer ce fichier   minisymposium-ABCDEF.tex
%        en rempla\c cant   ABCDEF   par l'acronyme de votre mini-symposium.
% *b* remplir et compl\`eter le fichier ainsi renomm\'e, suivant les indications ci-dessous.
% *c* au cours de ce travail, ne supprimer et  ne modifier aucun des macros ! merci.
% *d* compiler les fichiers SMAI-MS-Descriptions.tex  et  SMAI-MS-Resumes.tex
%        apr�s avoir remplac� dans   \input{Minisymposium-ABCDEF}
%        le mot ABCDEF par l'acronyme de votre mini-symposium ;
%        le r�sum�   SMAI-MS-Resumes.pdf   obtenu apr�s compilation ne devra pas d�passer une page,
%        et la description d�taill�e   SMAI-MS-Descriptions.pdf  obtenue apr�s compilation ne devra pas d�passer 5 pages.
% *e* merci de nous faire parvenir un document dans lequel le texlog ne pr�sente aucune erreur de compilation.


%% REMPLIR ICI L'ACRONYME
\renewcommand{\acronyme}{ABCDEF}


%% REMPLIR ICI LE TITRE
\titre{Du calcul symbolique � l'apprentissage, � bord du Python-Express}


%% INDIQUER ICI SI VOTRE MINI-SYMPOPSIUM EST PORTE PAR UNE ANR, UN GDR, OU UN AUTRE GROUPEMENT
%% (METTRE EN COMMENTAIRE (\%) SI NON UTILISE)
\support{Mini-symposium port� par le GdR Calcul}


%% REMPLIR ICI LE RESUME DU MINI-SYMPOSIUM
\Resume{

  L'objectif de ce mini-symposium est de donner un aper�u des
  fonctionnalit�s offertes par l'�cosyst�me des logiciels libres
  autour de Python pour les math�matiques pures et appliqu�es:
  mod�lisation, calcul interactif et haute performance, visualisation,
  traitement de donn�es, le tout accessible depuis un environnement
  interactif unifi�.

  Ce mini-symposium sera suivi dans la soir�e par un tutoriel
  informel, avec aide � l'installation, etc. Les jours suivants les
  organisateurs seront � la disposition des participants pour une aide
  plus personnalis�e.
}


%% REMPLIR ICI LA LISTE DES ORGANISATEURS
\ListeOrganisateurs{
  \organisateur{Paul}{Zimmermann}{Loria}
  \organisateur{Nicolas M.}{Thi�ry}{Laboratoire de Recherche en Informatique, Universit� Paris Sud}
}

%% REMPLIR ICI LA LISTE DES EXPOSES
\ListeExposes{
  \expose{Max}{Albert}{Computational Modelling Group, University of Southampton}{De la mod�lisation � la visualisation avec Jupyter/IPython}
  \expose{Nicolas M.}{Thi�ry}{Laboratoire de Recherche en Informatique, Universit� Paris Sud}{Alg�bre et math�matique discr�tes avec Sage}
  \expose{Lo�c}{Gouarin}{Laboratoire de Math�matique d'Orsay, Universit� Paris Sud}{Python performant avec Cython et Pythran}
  \expose{Fabien}{Pedegrosa}{INRIA Saclay-�le de France}{Apprentissage statistique avec scikit-learn}
}


%% REMPLIR ICI LES REF BIBLIOGRAPHIQUES DU RESUME (METTRE EN COMMENTAIRE (\%) SI NON UTILISE)
% \BiblioResume{
% 	\BiblioItem{ref1}{Auteur}{Titre du livre}{Editeur, ville, ann\'ee}
% 	\BiblioItem{ref2}{Auteur}{Titre de l'article}{Titre de la revue, volume et fascicule, ann\'ee, pages}
% }


%% REMPLIR ICI LES ADRESSES LONGUES DES ORGANISATEURS (et \'eventuellement des orateurs)
\ListeAdresses{
	\Adresse{Paul Zimmermann}{\url{www.loria.fr/~zimmerma/}}{Paul.Zimmermann(at)loria.fr}
	\Adresse{Max Albert}{\url{http://cmg.soton.ac.uk/people/mha2e09/}}{maximilian.albert(at)gmail.com}
	\Adresse{Nicolas M. Thi�ry}{\url{Nicolas.Thiery.name}}{Nicolas.Thiery(at)u-psud.fr}
	\Adresse{Lo�c Gouarin}{\url{http://www.math.u-psud.fr/~gouarin/}}{Loic.Gouarin(at)math.u-psud.fr}
	\Adresse{Fabien Pedegrosa}{\url{https://www.mendeley.com/profiles/fabian-pedregosa/}}{f(at)bianp.net}
}


%% REMPLIR ICI LA DESCRIPTION DU MS
\Description{

	% R\'ediger ici la description d\'etaill\'ee du mini-symposium et des interventions qui le compose.
	% Vous pouvez structurer votre description en sous-sections comme sugg\'er\'e ci-dessous.

	% \medskip

	% Le format est volontairement tr\`es libre, vous n'\^etes pas oblig\'e d'utiliser une section par pr\'esentation.

	% \medskip

	% Vous pouvez utiliser des r\'ef\'erences bibliographiques, \cite{ref1}, \cite{ref2}, \cite{ref3}, \cite{ref4}.


	%% SECTION D'INTRODUCTION %%
	%\section*{Introduction}

	% Vous pouvez ins\'erer des images (voir figure~\ref{fig:fig1}), en prenant soin de commencer le nom du fichier utilis\'e
	% (au format EPS) par l'acronyme de votre mini-symposium.

	% \medskip

	% Prenez soin \`a ce que le fichier PDF, une fois compil\'e, soit aussi l\'eger que possible (ne pas d\'epasser si possible 1Mo).

	% %% EXEMPLE DE FIGURE %%
	% \begin{figure}[ht]
	% 	\centering
	% 	\includegraphics[width=.4\textwidth]{ABCDEF-fig1}
	% 	\caption{\label{fig:fig1} Mettre ici une l\'egende.}
	% \end{figure}

	%% SECTION 1 %%
	\section{De la mod�lisation � la visualisation avec Jupyter/IPython}

	Bla bla.

	%% SECTION 2 %%
	\section{Titre de la deuxi\`eme section}

	Bla bla.

	%% SECTION 3 %%
	\section{Python performant avec Cython et Pythran}

	Bla bla.

	%% SECTION CONCLUSION %%
	\section*{Apprentissage statistique avec scikit-learn}

	Bla bla.
}


%% REMPLIR ICI LES REF BIBLIOGRAPHIQUES DE LA DESCRIPTION DETAILLEE (METTRE EN COMMENTAIRE (\%) SI NON UTILISE)
% \BiblioDescription{
% 	\BiblioItem{ref1}{Auteur}{Titre du livre}{Editeur, ville, ann\'ee}
% 	\BiblioItem{ref2}{Auteur}{Titre de l'article}{Titre de la revue, volume et fascicule, ann\'ee, pages}
% 	\BiblioItem{ref3}{Auteur}{Titre de l'article}{Titre de la revue, volume et fascicule, ann\'ee, pages}
% 	\BiblioItem{ref4}{Auteur}{Titre de l'article}{Titre de la revue, volume et fascicule, ann\'ee, pages}
% }

%% NE RIEN CHANGER EN DESSOUS
\MSFooter{}

%        le mot ABCDEF par l'acronyme de votre mini-symposium ;
%        le r�sum�   SMAI-MS-Resumes.pdf   obtenu apr�s compilation ne devra pas d�passer une page,
%        et la description d�taill�e   SMAI-MS-Descriptions.pdf  obtenue apr�s compilation ne devra pas d�passer 5 pages.
% *e* merci de nous faire parvenir un document dans lequel le texlog ne pr�sente aucune erreur de compilation.


%% REMPLIR ICI L'ACRONYME
\renewcommand{\acronyme}{ABCDEF}


%% REMPLIR ICI LE TITRE
\titre{Du calcul symbolique � l'apprentissage, � bord du Python-Express}


%% INDIQUER ICI SI VOTRE MINI-SYMPOPSIUM EST PORTE PAR UNE ANR, UN GDR, OU UN AUTRE GROUPEMENT
%% (METTRE EN COMMENTAIRE (\%) SI NON UTILISE)
\support{Mini-symposium port� par le GdR Calcul}


%% REMPLIR ICI LE RESUME DU MINI-SYMPOSIUM
\Resume{

  L'objectif de ce mini-symposium est de donner un aper�u des
  fonctionnalit�s offertes par l'�cosyst�me des logiciels libres
  autour de Python pour les math�matiques pures et appliqu�es:
  mod�lisation, calcul interactif et haute performance, visualisation,
  traitement de donn�es, le tout accessible depuis un environnement
  interactif unifi�.

  Ce mini-symposium sera suivi dans la soir�e par un tutoriel
  informel, avec aide � l'installation, etc. Les jours suivants les
  organisateurs seront � la disposition des participants pour une aide
  plus personnalis�e.
}


%% REMPLIR ICI LA LISTE DES ORGANISATEURS
\ListeOrganisateurs{
  \organisateur{Paul}{Zimmermann}{Loria}
  \organisateur{Nicolas M.}{Thi�ry}{Laboratoire de Recherche en Informatique, Universit� Paris Sud}
}

%% REMPLIR ICI LA LISTE DES EXPOSES
\ListeExposes{
  \expose{Max}{Albert}{Computational Modelling Group, University of Southampton}{De la mod�lisation � la visualisation avec Jupyter/IPython}
  \expose{Nicolas M.}{Thi�ry}{Laboratoire de Recherche en Informatique, Universit� Paris Sud}{Alg�bre et math�matique discr�tes avec Sage}
  \expose{Lo�c}{Gouarin}{Laboratoire de Math�matique d'Orsay, Universit� Paris Sud}{Python performant avec Cython et Pythran}
  \expose{Fabien}{Pedegrosa}{INRIA Saclay-�le de France}{Apprentissage statistique avec scikit-learn}
}


%% REMPLIR ICI LES REF BIBLIOGRAPHIQUES DU RESUME (METTRE EN COMMENTAIRE (\%) SI NON UTILISE)
% \BiblioResume{
% 	\BiblioItem{ref1}{Auteur}{Titre du livre}{Editeur, ville, ann\'ee}
% 	\BiblioItem{ref2}{Auteur}{Titre de l'article}{Titre de la revue, volume et fascicule, ann\'ee, pages}
% }


%% REMPLIR ICI LES ADRESSES LONGUES DES ORGANISATEURS (et \'eventuellement des orateurs)
\ListeAdresses{
	\Adresse{Paul Zimmermann}{\url{www.loria.fr/~zimmerma/}}{Paul.Zimmermann(at)loria.fr}
	\Adresse{Max Albert}{\url{http://cmg.soton.ac.uk/people/mha2e09/}}{maximilian.albert(at)gmail.com}
	\Adresse{Nicolas M. Thi�ry}{\url{Nicolas.Thiery.name}}{Nicolas.Thiery(at)u-psud.fr}
	\Adresse{Lo�c Gouarin}{\url{http://www.math.u-psud.fr/~gouarin/}}{Loic.Gouarin(at)math.u-psud.fr}
	\Adresse{Fabien Pedegrosa}{\url{https://www.mendeley.com/profiles/fabian-pedregosa/}}{f(at)bianp.net}
}


%% REMPLIR ICI LA DESCRIPTION DU MS
\Description{

	% R\'ediger ici la description d\'etaill\'ee du mini-symposium et des interventions qui le compose.
	% Vous pouvez structurer votre description en sous-sections comme sugg\'er\'e ci-dessous.

	% \medskip

	% Le format est volontairement tr\`es libre, vous n'\^etes pas oblig\'e d'utiliser une section par pr\'esentation.

	% \medskip

	% Vous pouvez utiliser des r\'ef\'erences bibliographiques, \cite{ref1}, \cite{ref2}, \cite{ref3}, \cite{ref4}.


	%% SECTION D'INTRODUCTION %%
	%\section*{Introduction}

	% Vous pouvez ins\'erer des images (voir figure~\ref{fig:fig1}), en prenant soin de commencer le nom du fichier utilis\'e
	% (au format EPS) par l'acronyme de votre mini-symposium.

	% \medskip

	% Prenez soin \`a ce que le fichier PDF, une fois compil\'e, soit aussi l\'eger que possible (ne pas d\'epasser si possible 1Mo).

	% %% EXEMPLE DE FIGURE %%
	% \begin{figure}[ht]
	% 	\centering
	% 	\includegraphics[width=.4\textwidth]{ABCDEF-fig1}
	% 	\caption{\label{fig:fig1} Mettre ici une l\'egende.}
	% \end{figure}

	%% SECTION 1 %%
	\section{De la mod�lisation � la visualisation avec Jupyter/IPython}

	Bla bla.

	%% SECTION 2 %%
	\section{Titre de la deuxi\`eme section}

	Bla bla.

	%% SECTION 3 %%
	\section{Python performant avec Cython et Pythran}

	Bla bla.

	%% SECTION CONCLUSION %%
	\section*{Apprentissage statistique avec scikit-learn}

	Bla bla.
}


%% REMPLIR ICI LES REF BIBLIOGRAPHIQUES DE LA DESCRIPTION DETAILLEE (METTRE EN COMMENTAIRE (\%) SI NON UTILISE)
% \BiblioDescription{
% 	\BiblioItem{ref1}{Auteur}{Titre du livre}{Editeur, ville, ann\'ee}
% 	\BiblioItem{ref2}{Auteur}{Titre de l'article}{Titre de la revue, volume et fascicule, ann\'ee, pages}
% 	\BiblioItem{ref3}{Auteur}{Titre de l'article}{Titre de la revue, volume et fascicule, ann\'ee, pages}
% 	\BiblioItem{ref4}{Auteur}{Titre de l'article}{Titre de la revue, volume et fascicule, ann\'ee, pages}
% }

%% NE RIEN CHANGER EN DESSOUS
\MSFooter{}

%        le mot ABCDEF par l'acronyme de votre mini-symposium ;
%        le r�sum�   SMAI-MS-Resumes.pdf   obtenu apr�s compilation ne devra pas d�passer une page,
%        et la description d�taill�e   SMAI-MS-Descriptions.pdf  obtenue apr�s compilation ne devra pas d�passer 5 pages.
% *e* merci de nous faire parvenir un document dans lequel le texlog ne pr�sente aucune erreur de compilation.


%% REMPLIR ICI L'ACRONYME
\renewcommand{\acronyme}{ABCDEF}


%% REMPLIR ICI LE TITRE
\titre{Du calcul symbolique � l'apprentissage, � bord du Python-Express}


%% INDIQUER ICI SI VOTRE MINI-SYMPOPSIUM EST PORTE PAR UNE ANR, UN GDR, OU UN AUTRE GROUPEMENT
%% (METTRE EN COMMENTAIRE (\%) SI NON UTILISE)
\support{Mini-symposium port� par le GdR Calcul}


%% REMPLIR ICI LE RESUME DU MINI-SYMPOSIUM
\Resume{

  L'objectif de ce mini-symposium est de donner un aper�u des
  fonctionnalit�s offertes par l'�cosyst�me des logiciels libres
  autour de Python pour les math�matiques pures et appliqu�es:
  mod�lisation, calcul interactif et haute performance, visualisation,
  traitement de donn�es, le tout accessible depuis un environnement
  interactif unifi�.

  Ce mini-symposium sera suivi dans la soir�e par un tutoriel
  informel, avec aide � l'installation, etc. Les jours suivants les
  organisateurs seront � la disposition des participants pour une aide
  plus personnalis�e.
}


%% REMPLIR ICI LA LISTE DES ORGANISATEURS
\ListeOrganisateurs{
  \organisateur{Paul}{Zimmermann}{Loria}
  \organisateur{Nicolas M.}{Thi�ry}{Laboratoire de Recherche en Informatique, Universit� Paris Sud}
}

%% REMPLIR ICI LA LISTE DES EXPOSES
\ListeExposes{
  \expose{Max}{Albert}{Computational Modelling Group, University of Southampton}{De la mod�lisation � la visualisation avec Jupyter/IPython}
  \expose{Nicolas M.}{Thi�ry}{Laboratoire de Recherche en Informatique, Universit� Paris Sud}{Alg�bre et math�matique discr�tes avec Sage}
  \expose{Lo�c}{Gouarin}{Laboratoire de Math�matique d'Orsay, Universit� Paris Sud}{Python performant avec Cython et Pythran}
  \expose{Fabien}{Pedegrosa}{INRIA Saclay-�le de France}{Apprentissage statistique avec scikit-learn}
}


%% REMPLIR ICI LES REF BIBLIOGRAPHIQUES DU RESUME (METTRE EN COMMENTAIRE (\%) SI NON UTILISE)
% \BiblioResume{
% 	\BiblioItem{ref1}{Auteur}{Titre du livre}{Editeur, ville, ann\'ee}
% 	\BiblioItem{ref2}{Auteur}{Titre de l'article}{Titre de la revue, volume et fascicule, ann\'ee, pages}
% }


%% REMPLIR ICI LES ADRESSES LONGUES DES ORGANISATEURS (et \'eventuellement des orateurs)
\ListeAdresses{
	\Adresse{Paul Zimmermann}{\url{www.loria.fr/~zimmerma/}}{Paul.Zimmermann(at)loria.fr}
	\Adresse{Max Albert}{\url{http://cmg.soton.ac.uk/people/mha2e09/}}{maximilian.albert(at)gmail.com}
	\Adresse{Nicolas M. Thi�ry}{\url{Nicolas.Thiery.name}}{Nicolas.Thiery(at)u-psud.fr}
	\Adresse{Lo�c Gouarin}{\url{http://www.math.u-psud.fr/~gouarin/}}{Loic.Gouarin(at)math.u-psud.fr}
	\Adresse{Fabien Pedegrosa}{\url{https://www.mendeley.com/profiles/fabian-pedregosa/}}{f(at)bianp.net}
}


%% REMPLIR ICI LA DESCRIPTION DU MS
\Description{

	% R\'ediger ici la description d\'etaill\'ee du mini-symposium et des interventions qui le compose.
	% Vous pouvez structurer votre description en sous-sections comme sugg\'er\'e ci-dessous.

	% \medskip

	% Le format est volontairement tr\`es libre, vous n'\^etes pas oblig\'e d'utiliser une section par pr\'esentation.

	% \medskip

	% Vous pouvez utiliser des r\'ef\'erences bibliographiques, \cite{ref1}, \cite{ref2}, \cite{ref3}, \cite{ref4}.


	%% SECTION D'INTRODUCTION %%
	%\section*{Introduction}

	% Vous pouvez ins\'erer des images (voir figure~\ref{fig:fig1}), en prenant soin de commencer le nom du fichier utilis\'e
	% (au format EPS) par l'acronyme de votre mini-symposium.

	% \medskip

	% Prenez soin \`a ce que le fichier PDF, une fois compil\'e, soit aussi l\'eger que possible (ne pas d\'epasser si possible 1Mo).

	% %% EXEMPLE DE FIGURE %%
	% \begin{figure}[ht]
	% 	\centering
	% 	\includegraphics[width=.4\textwidth]{ABCDEF-fig1}
	% 	\caption{\label{fig:fig1} Mettre ici une l\'egende.}
	% \end{figure}

	%% SECTION 1 %%
	\section{De la mod�lisation � la visualisation avec Jupyter/IPython}

	Bla bla.

	%% SECTION 2 %%
	\section{Titre de la deuxi\`eme section}

	Bla bla.

	%% SECTION 3 %%
	\section{Python performant avec Cython et Pythran}

	Bla bla.

	%% SECTION CONCLUSION %%
	\section*{Apprentissage statistique avec scikit-learn}

	Bla bla.
}


%% REMPLIR ICI LES REF BIBLIOGRAPHIQUES DE LA DESCRIPTION DETAILLEE (METTRE EN COMMENTAIRE (\%) SI NON UTILISE)
% \BiblioDescription{
% 	\BiblioItem{ref1}{Auteur}{Titre du livre}{Editeur, ville, ann\'ee}
% 	\BiblioItem{ref2}{Auteur}{Titre de l'article}{Titre de la revue, volume et fascicule, ann\'ee, pages}
% 	\BiblioItem{ref3}{Auteur}{Titre de l'article}{Titre de la revue, volume et fascicule, ann\'ee, pages}
% 	\BiblioItem{ref4}{Auteur}{Titre de l'article}{Titre de la revue, volume et fascicule, ann\'ee, pages}
% }

%% NE RIEN CHANGER EN DESSOUS
\MSFooter{}

%        le mot ABCDEF par l'acronyme de votre mini-symposium ;
%        le r�sum�   SMAI-MS-Resumes.pdf   obtenu apr�s compilation ne devra pas d�passer une page,
%        et la description d�taill�e   SMAI-MS-Descriptions.pdf  obtenue apr�s compilation ne devra pas d�passer 5 pages.
% *e* merci de nous faire parvenir un document dans lequel le texlog ne pr�sente aucune erreur de compilation.


%% REMPLIR ICI L'ACRONYME
\renewcommand{\acronyme}{Python}


%% REMPLIR ICI LE TITRE
\titre{Du calcul symbolique � l'apprentissage, � bord du Python-Express}


%% INDIQUER ICI SI VOTRE MINI-SYMPOPSIUM EST PORTE PAR UNE ANR, UN GDR, OU UN AUTRE GROUPEMENT
%% (METTRE EN COMMENTAIRE (\%) SI NON UTILISE)
\support{Mini-symposium port� par le GdR Calcul}


%% REMPLIR ICI LE RESUME DU MINI-SYMPOSIUM
\Resume{

  L'objectif de ce mini-symposium est de donner un aper�u des
  fonctionnalit�s offertes par l'�cosyst�me des logiciels libres
  autour de Python pour les math�matiques pures et appliqu�es:
  mod�lisation, calcul interactif et haute performance, visualisation,
  traitement de donn�es, le tout accessible depuis un environnement
  interactif unifi�.

  Ce mini-symposium sera suivi dans la soir�e par un tutoriel
  informel, avec aide � l'installation, etc. Les jours suivants les
  organisateurs seront � la disposition des participants pour une aide
  plus personnalis�e.
}


%% REMPLIR ICI LA LISTE DES ORGANISATEURS
\ListeOrganisateurs{
  \organisateur{Paul}{Zimmermann}{Loria}
  \organisateur{Nicolas M.}{Thi�ry}{Laboratoire de Recherche en Informatique, Universit� Paris Sud}
}

%% REMPLIR ICI LA LISTE DES EXPOSES
\ListeExposes{
  \expose{Maximilian}{Albert}{Computational Modelling Group, University of Southampton}{De la mod�lisation � la visualisation avec Jupyter/IPython}
  \expose{Nicolas M.}{Thi�ry}{Laboratoire de Recherche en Informatique, Universit� Paris Sud}{Alg�bre et math�matique discr�tes avec Sage}
  \expose{Lo�c}{Gouarin}{Laboratoire de Math�matique d'Orsay, Universit� Paris Sud}{Python performant avec Cython et Pythran}
  \expose{Fabien}{Pedegrosa}{INRIA Saclay-�le de France}{Apprentissage statistique avec scikit-learn}
}


%% REMPLIR ICI LES REF BIBLIOGRAPHIQUES DU RESUME (METTRE EN COMMENTAIRE (\%) SI NON UTILISE)
% \BiblioResume{
% 	\BiblioItem{ref1}{Auteur}{Titre du livre}{Editeur, ville, ann\'ee}
% 	\BiblioItem{ref2}{Auteur}{Titre de l'article}{Titre de la revue, volume et fascicule, ann\'ee, pages}
% }


%% REMPLIR ICI LES ADRESSES LONGUES DES ORGANISATEURS (et \'eventuellement des orateurs)
\ListeAdresses{
	\Adresse{Paul Zimmermann}{\url{www.loria.fr/~zimmerma/}}{Paul.Zimmermann(at)loria.fr}
	\Adresse{Maximilian Albert}{\url{http://cmg.soton.ac.uk/people/mha2e09/}}{maximilian.albert(at)gmail.com}
	\Adresse{Nicolas M. Thi�ry}{\url{Nicolas.Thiery.name}}{Nicolas.Thiery(at)u-psud.fr}
	\Adresse{Lo�c Gouarin}{\url{http://www.math.u-psud.fr/~gouarin/}}{Loic.Gouarin(at)math.u-psud.fr}
	\Adresse{Fabien Pedegrosa}{\url{https://www.mendeley.com/profiles/fabian-pedregosa/}}{f(at)bianp.net}
}


%% REMPLIR ICI LA DESCRIPTION DU MS
\Description{

	% R\'ediger ici la description d\'etaill\'ee du mini-symposium et des interventions qui le compose.
	% Vous pouvez structurer votre description en sous-sections comme sugg\'er\'e ci-dessous.

	% \medskip

	% Le format est volontairement tr\`es libre, vous n'\^etes pas oblig\'e d'utiliser une section par pr\'esentation.

	% \medskip

	% Vous pouvez utiliser des r\'ef\'erences bibliographiques, \cite{ref1}, \cite{ref2}, \cite{ref3}, \cite{ref4}.


	%% SECTION D'INTRODUCTION %%
	%\section*{Introduction}

	% Vous pouvez ins\'erer des images (voir figure~\ref{fig:fig1}), en prenant soin de commencer le nom du fichier utilis\'e
	% (au format EPS) par l'acronyme de votre mini-symposium.

	% \medskip

	% Prenez soin \`a ce que le fichier PDF, une fois compil\'e, soit aussi l\'eger que possible (ne pas d\'epasser si possible 1Mo).

	% %% EXEMPLE DE FIGURE %%
	% \begin{figure}[ht]
	% 	\centering
	% 	\includegraphics[width=.4\textwidth]{ABCDEF-fig1}
	% 	\caption{\label{fig:fig1} Mettre ici une l\'egende.}
	% \end{figure}

	%% SECTION 1 %%
	\section{De la mod�lisation � la visualisation avec Jupyter/IPython}

	Bla bla.

	%% SECTION 2 %%
	\section{Titre de la deuxi\`eme section}

	Bla bla.

	%% SECTION 3 %%
	\section{Python performant avec Cython et Pythran}

	Bla bla.

	%% SECTION CONCLUSION %%
	\section*{Apprentissage statistique avec scikit-learn}

	Bla bla.
}


%% REMPLIR ICI LES REF BIBLIOGRAPHIQUES DE LA DESCRIPTION DETAILLEE (METTRE EN COMMENTAIRE (\%) SI NON UTILISE)
% \BiblioDescription{
% 	\BiblioItem{ref1}{Auteur}{Titre du livre}{Editeur, ville, ann\'ee}
% 	\BiblioItem{ref2}{Auteur}{Titre de l'article}{Titre de la revue, volume et fascicule, ann\'ee, pages}
% 	\BiblioItem{ref3}{Auteur}{Titre de l'article}{Titre de la revue, volume et fascicule, ann\'ee, pages}
% 	\BiblioItem{ref4}{Auteur}{Titre de l'article}{Titre de la revue, volume et fascicule, ann\'ee, pages}
% }

%% NE RIEN CHANGER EN DESSOUS
\MSFooter{}
